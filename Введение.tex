\section*{ВВЕДЕНИЕ}
\addcontentsline{toc}{section}{ВВЕДЕНИЕ}

Современные информационные технологии играют ключевую роль в развитии бизнеса, особенно в сфере розничной торговли. Интернет-магазины стали важным инструментом для взаимодействия с клиентами, предоставляя удобный доступ к товарам и услугам в любое время и из любой точки мира. В книжной индустрии онлайн-торговля позволяет не только расширить ассортимент, но и автоматизировать процессы управления, упрощая работу как для покупателей, так и для сотрудников магазина. Разработка веб-приложений для таких задач требует применения современных технологий, обеспечивающих надежность, масштабируемость и удобство использования.

Создание программно-информационной системы для управления книжным магазином позволяет оптимизировать процессы обработки заказов, ведения каталога книг и взаимодействия с клиентами. Такая система предоставляет покупателям возможность просматривать книги, оформлять заказы и отслеживать их статус, а сотрудникам и администраторам — эффективно управлять ассортиментом и данными пользователей.

\emph{Цель настоящей работы} – разработка веб-приложения для автоматизации управления книжным магазином, обеспечивающего удобный интерфейс для клиентов и функционал для администрирования. Для достижения цели необходимо решить \emph{следующие задачи:}
\begin{itemize}
\item провести анализ предметной области интернет-торговли книгами;
\item разработать концептуальную модель системы управления книжным магазином;
\item спроектировать программную систему;
\item реализовать и протестировать веб-приложение с использованием веб-технологий.
\end{itemize}

\emph{Структура и объем работы.} Отчет состоит из введения, 4 разделов основной части, заключения, списка использованных источников, 2 приложений. Текст выпускной квалификационной работы равен \formbytotal{lastpage}{страниц}{е}{ам}{ам}.

\emph{Во введении} сформулирована цель работы, поставлены задачи разработки, описана структура работы, приведено краткое содержание каждого из разделов.

\emph{В первом разделе} на стадии описания технической характеристики предметной области приводится анализ предметной области, включая особенности интернет-торговли книгами и требования к системе.

\emph{Во втором разделе} на стадии технического задания приводятся требования к разрабатываемой системе.

\emph{В третьем разделе} на стадии технического проектирования представлен технический проект, включая выбор технологий, проектирование архитектуры и пользовательского интерфейса.

\emph{В четвертом разделе} приводится список модулей и их методов, использованных при разработке системы, производится тестирование разработанного сайта.

В заключении излагаются основные результаты работы, полученные в ходе разработки.

В приложении А представлен графический материал.
В приложении Б представлены фрагменты исходного кода. 
