\addcontentsline{toc}{section}{СПИСОК ИСПОЛЬЗОВАННЫХ ИСТОЧНИКОВ}

\begin{thebibliography}{9}

\bibitem{1}
Манина А.И. Понятие и особенности правового регулирования торговли в сети Интернет // Вестник науки. – 2021. – № 1 (34), т. 3. – С. 126–130. – ISSN: 27128849. – URL: https://www.xn----8sbempclcwd3bmt.xn--p1ai/article/4026.
\bibitem{2}
Корнева С.С. Развитие электронной торговли в Российской Федерации // Известия ОГАУ. – 2011. – № 32-1. – С. 208–211. – URL: https://cyberleninka.ru/article/n/razvitie-elektronnoy-torgovli-v-rossiyskoy-federatsii.
\bibitem{3}
Статья 26.1. Дистанционный способ продажи товара // КонсультантПлюс. – URL: https://www.consultant.ru/document/cons\_doc\_LAW\_
305/1525b1a2f037db240c8e6a749619f86e53857f13.
\bibitem{4}
Вафина Н.Х. Интернет-магазин // Мир новой экономики. – 2010. – № 2. – С. 28–32. – EDN: ZEZNAH.
\bibitem{5}
Тагавердиева Д.С. Интернет-торговля как фактор развития бизнеса // УЭПС: экономика, политика, право. – 2021. – № 2. – С. 5–12. – URL: https://cyberleninka.ru/article/n/internet-torgovlya-kak-faktor-razvitiya-biznesa.
\bibitem{6}
Новостной портал «Newsland». – URL: https://newsland.com/news/detail/id/1007025.
\bibitem{7}
Воскресенская О.В. Влияние интернет-магазинов на покупательскую способность потребителей // E-Scio. – 2023. – № 2 (77). – С. 45–52. – URL: https://cyberleninka.ru/article/n/vliyanie-internet-magazinov-na-pokupatelskuyu-sposobnost-potrebiteley.
\bibitem{8}
Чепелева А.Ю., Хайрхуа А.Ф. Интернет-торговля в России как основной элемент цифровой экономики // Студенческий научный форум: материалы XI Международной студенческой научной конференции, 2019. – С. 1–5. – URL: https://scienceforum.ru/2019/article/2018010575.
\bibitem{9}
Косников С.Н., Королёв Д.А., Чивви Е.Н. и др. История электронной коммерции в России и за рубежом // ЕГИ. – 2022. – № 6 (44). – С. 33–40. – URL: https://cyberleninka.ru/article/n/istoriya-elektronnoy-kommertsii-v-rossii-i-za-rubezhom.
\bibitem{10}
Data Insight. Исследование DI eCommerce 2020. – URL: https://datainsight.ru/DI\_eCommerce2020.
\bibitem{11}
Хамдохова Х.Р., Кяова А.А., Шагиров А.С. и др. Влияние пандемии COVID-19 на российский рынок интернет-торговли // Авангард молодежной науки: сборник статей II Международного научно-исследовательского конкурса, Петрозаводск, 28 марта 2022 года. – Петрозаводск: МЦНП «Новая Наука», 2022. – С. 208–212. – EDN: KWHZHR.
\bibitem{12}
Пушкарева Е.П. Книжная интернет-торговля: зарубежный опыт и российские реалии // Вестник МГУП. – 2011. – № 3. – С. 15–22. – URL: https://cyberleninka.ru/article/n/knizhnaya-internet-torgovlya-zarubezhnyy-opyt-i-rossiyskie-realii.
\bibitem{13}
Мандел Т. Разработка пользовательского интерфейса. – Москва: ДМК Пресс, 2019. – 420 с. – ISBN: 9785041950606.
\bibitem{14}
Фримен А. Практикум по программированию на JavaScript. – Москва: Вильямс, 2013. – 960 с. – ISBN: 9785845917997.
\bibitem{15}
Буч Г., Якобсон И., Рамбо Д. Введение в UML от создателей языка. – Москва: ДМК Пресс, 2015. – 498 с. – ISBN: 9785457433793.
\end{thebibliography}
