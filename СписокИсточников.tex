\addcontentsline{toc}{section}{СПИСОК ИСПОЛЬЗОВАННЫХ ИСТОЧНИКОВ}

\begin{thebibliography}{9}

    \bibitem{1} Манина А.И. Понятие и особенности правового регулирования торговли в сети Интернет // Вестник науки. – 2021. – №1 (34), т. 3. – С. 126–130. – ISSN 2712-8849. – URL: https://www.xn----8sbempclcwd3bmt.xn--p1ai/article/4026 (дата обращения: 13.05.2025 г.). Текст~: непосредственный.
    \bibitem{2} Корнева Светлана Сагитовна Развитие электронной торговли в Российской Федерации // Известия ОГАУ. 2011. №32-1. URL: https://cyberleninka.ru/article/n/razvitie-elektronnoy-torgovli-v-rossiyskoy-federatsii (дата обращения: 14.05.2025). Текст~: непосредственный.
    \bibitem{3} Статья 26.1. Дистанционный способ продажи товара // КонсультантПлюс [Электронный ресурс]. – URL: https://www.consultant.ru/document/cons\_doc\_LAW\_305/1525b1a2f037db240c8e6a749619f86e53857f13/ (дата обращения: 03.05.2024 г.). Текст~: непосредственный.
    \bibitem{4} Вафина Н.Х. Интернет-магазин // Мир новой экономики. – 2010. – №2. – С. 28–32. – EDN ZEZNAH. Текст~: непосредственный.
	\bibitem{5} Тагавердиева Д.С. Интернет-торговля как фактор развития бизнеса // УЭПС. – 2021. – №2. – URL: https://cyberleninka.ru/article/n/internet-torgovlya-kak-faktor-razvitiya-biznesa (дата обращения: 13.05.2025 г.). Текст~: непосредственный.
	\bibitem{6} Новостной портал «Newsland» [Электронный ресурс]. – URL: http://newsland.com/news/detail/id/1007025. Текст~: непосредственный.
	\bibitem{7} Воскресенская О.В. Влияние интернет-магазинов на покупательскую способность потребителей // E-Scio. – 2023. – №2 (77). – URL: https://cyberleninka.ru/article/n/vliyanie-internet-magazinov-na-pokupatelskuyu-sposobnost-potrebiteley (дата обращения: 13.05.2025 г.). Текст~: непосредственный.
	\bibitem{8} Чепелева А.Ю., Хайрхуа А.Ф. Интернет-торговля в России как основной элемент цифровой экономики // Материалы XI Международной студенческой научной конференции «Студенческий научный форум». – URL: https://scienceforum.ru/2019/article/2018010575. Текст~: непосредственный.
	\bibitem{9} Косников С.Н., Королёв Д.А., Чивви Е.Н., Разумова Д.Ю. История электронной коммерции в России и за рубежом // ЕГИ. – 2022. – №6 (44). – URL: https://cyberleninka.ru/article/n/istoriya-elektronnoy-kommertsii-v-rossii-i-za-rubezhom (дата обращения: 13.05.2025 г.). Текст~: непосредственный.
	\bibitem{10} Data Insight. Исследование DI eCommerce 2020 [Электронный ресурс]. – URL: https://datainsight.ru/DI\_eCommerce2020. Текст~: непосредственный.
	\bibitem{11} Влияние пандемии COVID-19 на российский рынок интернет-торговли / Х.Р. Хамдохова, А.А. Кяова, А.С. Шагиров [и др.] // Авангард молодежной науки: сборник статей II Международного научно-исследовательского конкурса, Петрозаводск, 28 марта 2022 года. – Петрозаводск: Международный центр научного партнерства «Новая Наука» (ИП Ивановская И.И.), 2022. – С. 208–212. – EDN KWHZHR. Текст~: непосредственный.
	\bibitem{12} Пушкарева Е.П. Книжная интернет-торговля: зарубежный опыт и российские реалии // Вестник МГУП. – 2011. – №3. – URL: https://cyberleninka.ru/article/n/knizhnaya-internet-torgovlya-zarubezhnyy-opyt-i-rossiyskie-realii (дата обращения: 13.05.2025 г.).  Текст~: непосредственный.
   
	
\end{thebibliography}
