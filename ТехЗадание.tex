\section{Техническое задание}
\subsection{Основание для разработки}

Основанием для разработки является задание на выпускную квалификационную работу бакалавра.

 Полное наименование системы: «Программно-информационная система управления книжным магазином».

\subsection{Цель и назначение разработки}

Разрабатываемая программно-информационная система предназначена для управления ассортиментом книг, обработки заказов, предоставления удобного интерфейса покупателям и сотрудникам.

Программа ориентирована на предпринимателей, планирующих в будущем выходить на всероссийский рынок.

Задачами данной разработки являются:
\begin{enumerate}
\item Создание серверной части на Flask для обработки запросов и взаимодействия с базой данных PostgreSQL.
\item Разработка клиентской части на HTML, CSS и JavaScript для отображения книг, корзины и управления заказами.
\item Реализация системы аутентификации и авторизации пользователей с разными ролями (клиент, сотрудник, администратор).
\item Интеграция механизмов поиска и пагинации для удобного просмотра каталога книг.
\end{enumerate}


\subsection{Требования к программной системе}


\subsubsection{Требования к данным программной системы}


На рисунке 2.1 представлена концептуальная модель базы данных данных программной системы в виде UML-диаграммы сущность-связь.


\begin{figure}[H]
	\centering
	\includegraphics[width=0.7\linewidth]{"images/Модель_данных"}
	\caption{Концептуальная модель базы данных}
	\label{fig:--}
\end{figure}

Входными данными системы являются:
\begin{itemize}
	\item сведения о книгах;
	\item параметры корзины;
	\item учётные данные пользователя.
\end{itemize}

Выходными данными являются:
\begin{itemize}
	\item идентификаторы оформленных заказов;
	\item уведомления о статусе заказов.
\end{itemize}



Программно-информационная система должна обеспечивать хранение и обновление данных о книгах, пользователях и заказах.

\subsubsection{Функциональные требования к программной системе}

Разрабатываемая программно-информационная система книжного магазина предусматривает три роли пользователей: покупатели, сотрудники и администраторы. Каждой роли доступны определенные функции.

Для всех пользователей:

\begin{enumerate}
	\item Просмотр каталога книг: отображение списка книг с возможностью сортировки и фильтрации, поиск книг по названию, автору или жанру.
	\item Просмотр деталей книги: информация о названии, авторе, цене, наличии на складе и описании.
	\item Авторизация и регистрация: вход в систему под своей учетной записью, создание нового аккаунта.
\end{enumerate}

Для зарегистрированных покупателей:
\begin{enumerate}
	\item Оформление заказов: подтверждение заказа с указанием итоговой суммы.
	\item Просмотр истории заказов: отслеживание своих заказов и их текущего статуса.
\end{enumerate}

Для сотрудников:
\begin{enumerate}
	\item Управление книгами: добавление новых книг в каталог, редактирование информации о существующих книгах, удаление книг из каталога.
	\item Управление заказами: просмотр всех заказов, изменение статуса заказов.
\end{enumerate}

Для администраторов:
\begin{enumerate}
	\item Управление пользователями: назначение ролей, просмотр списка всех пользователей.
	\item Полный доступ к функциям сотрудников: все возможности сотрудников, включая управление книгами и заказами.
\end{enumerate}

На рисунке 2.2 в виде диаграммы прецедентов представлены функциональные требования к системе, доступные для всех категорий пользователей.

На рисунке 2.3 представлены дополнительные функциональные требования к системе для авторизованных пользователей.

На рисунке 2.4 представлены дополнительные функциональные требования к системе для пользователей с ролью сотрудника.

На рисунке 2.5 представлены дополнительные функциональные требования к системе для пользователей с ролью администратора.

\begin{figure}[H]
	\centering
	\includegraphics[width=0.7\linewidth]{"images/Все_пользователи"}
	\caption{Диаграмма прецедентов для неавторизованных пользователей}
	\label{fig:--}
\end{figure}

\begin{figure}[H]
	\centering
	\includegraphics[width=0.7\linewidth]{"images/Авторизованные"}
	\caption{Диаграмма прецедентов для авторизованных пользователей}
	\label{fig:--}
\end{figure}

\begin{figure}[H]
	\centering
	\includegraphics[width=0.7\linewidth]{"images/Сотрудники"}
	\caption{Диаграмма прецедентов для пользователей с ролью сотрудника}
	\label{fig:--}
\end{figure}

\begin{figure}[H]
	\centering
	\includegraphics[width=0.7\linewidth]{"images/Администраторы"}
	\caption{Диаграмма прецедентов для пользователей с ролью администратора}
	\label{fig:--}
\end{figure}


\paragraph{Вариант использования «Просмотр каталога книг»}

Заинтересованные лица и их требования: пользователь желает ознакомиться с ассортиментом книг интернет-магазина.

Предусловие: открыта главная страница сайта.

Постусловие: пользователь видит список книг.

Основной успешный сценарий:

\begin{enumerate}
	\item Пользователь заходит на сайт.
	\item Приложение загружает список книг с пагинацией (по 8 книг на страницу).
\end{enumerate}


\paragraph{Вариант использования «Поиск книг»}

Заинтересованные лица и их требования: пользователь желает найти конкретную книгу.

Предусловие: открыта главная страница сайта.

Постусловие: пользователь находит или не находит нужную ему книгу.

Основной успешный сценарий:

\begin{enumerate}
	\item Пользователь пишет полное название книги или её часть в поле поиска.
	\item Приложение отправляет запрос к API. 
	\item Сервер возвращает книги, название которых соответствуют поисковому запросу.
	\item Пользователь может переключать страницы с результатами.
\end{enumerate}


\paragraph{Вариант использования «Просмотр деталей книги» }

Заинтересованные лица и их требования: пользователь, желает узнать подробности о книге.

Предусловие: открыта главная страница сайта.

Постусловие: пользователь видит полную информацию о книге.

Основной успешный сценарий:

\begin{enumerate}
	\item Пользователь нажимает на кнопку "Подробнее".
	\item Приложение отправляет запрос к API для получения полной информации о книге.
	\item Открывается модальное окно с деталями: автор, цена, описание, жанры и т.д.
	\item Пользователь может закрыть окно или добавить книгу в корзину.
\end{enumerate}


\paragraph{Вариант использования «Добавление книги в корзину авторизованным пользователем»}

Заинтересованные лица и их требования: пользователь, который авторизован на сайте, желает добавить книгу в корзину.

Предусловие: пользователь авторизован и открыта главная страница сайта.

Постусловие: книга добавлена в корзину.

Основной успешный сценарий:

\begin{enumerate}
	\item Пользователь нажимает кнопку "В корзину" на карточке книги.
	\item Если пользователь не авторизован, появляется уведомление с предложением войти.
	\item Книга добавляется в корзину (локальное хранилище и состояние UI обновляются).
	\item Кнопка меняется на "В корзине" и становится неактивной.
	\item Появляется возможность нажать кнопку "Оформить заказ" и отображается стоимость корзины.
\end{enumerate}

\paragraph{Вариант использования «Управление корзиной»}

Заинтересованные лица и их требования: авторизованный пользователь желает изменить свою корзину.

Предусловие: в корзине есть хотя бы один товар и открыта главная страница сайта.

Постусловие: состояние корзины обновлено.

Основной успешный сценарий:

\begin{enumerate}
	\item Пользователь спускается в корзину.
	\item Пользователь может увеличить или уменьшить количество выбранных книг.
	\item Пользователь может выборочно удалить книги из корзины.
	\item Обновляется общая стоимость заказа.
	\item Если товара нет в достаточном количестве, отображается предупреждение.
	\item При соблюдении условий становится доступна кнопка "Оформить заказ".
\end{enumerate}


\paragraph{Вариант использования «Оформление заказа»}

Заинтересованные лица и их требования: авторизованный покупатель, который желает заказать книги из корзины.

Предусловие: открыта главная страница сайта, есть книги в корзине и выбрано имеющееся на складе их количество.

Постусловие: заказ создан, корзина очищена.

Основной успешный сценарий:

\begin{enumerate}
	\item Пользователь нажимает "Оформить заказ".
	\item Приложение отправляет на сервер данные о заказе.
	\item Сервер проверяет наличие товаров и создает заказ.
	\item Пользователь получает уведомление об успешном оформлении с номером заказа и деталями.
	\item Корзина очищается.
\end{enumerate}


\paragraph{Вариант использования «Просмотр истории заказов»}

Заинтересованные лица и их требования: авторизованный покупатель, желает посмотреть на историю заказов.

Предусловие: пользователь авторизован и открыта главная страница сайта

Постусловие: отображается список заказов.

Основной успешный сценарий:

\begin{enumerate}
	\item Пользователь нажимает "Мои заказы".
	\item Приложение загружает список заказов через API.
	\item При наличии заказов для каждого отображаются номер, дата, статус, сумма.
\end{enumerate}


\paragraph{Вариант использования «Просмотр подробностей заказа»}

Заинтересованные лица и их требования: авторизованный покупатель желает подробнее изучить свой заказ.

Предусловие: пользователь авторизован, на главной странице и имеет хотя бы один заказ.

Постусловие: пользователь узнаёт подробности своего заказа.

Основной успешный сценарий:

\begin{enumerate}
	\item Пользователь нажимает "Мои заказы".
	\item Приложение загружает список заказов через API.
	\item Пользователь нажимает на один из своих заказов.
	\item При наличии этих книг в системе пользователь увидит сколько он заказал определённых книг и их стоимость по отдельности.
\end{enumerate}


\paragraph{Вариант использования «Просмотр истории заказов сотрудником или администратором»}

Заинтересованные лица и их требования: пользователь авторизованный на аккаунте с ролью сотрудника или администратора желает посмотреть список заказов всех пользователей.

Предусловие: пользователь авторизован, его аккаунт имеет роль сотрудника или администратора и открыта главная страница сайта.

Постусловие: пользователь видит список всех заказов.

Основной успешный сценарий:

\begin{enumerate}
	\item Пользователь нажимает "Все заказы".
	\item Приложение загружает список заказов через API.
	\item При наличии заказов для каждого отображаются номер, дата, статус, сумма.
\end{enumerate}


\paragraph{Вариант использования «Просмотр подробностей заказа сотрудником или администратором»}

Заинтересованные лица и их требования: пользователь авторизованный на аккаунте с ролью сотрудника или администратора желает посмотреть подробности заказа одного из пользователей.

Предусловие: пользователь авторизован, его аккаунт имеет роль сотрудника или администратора и открыта главная страница сайта.

Постусловие: пользователь узнаёт подробности заказа одного из пользователей.

Основной успешный сценарий:

\begin{enumerate}
	\item Пользователь нажимает "Все заказы".
	\item Приложение загружает список заказов через API.
	\item Пользователь нажимает на один из заказов.
	\item При наличии этих книг в системе пользователь увидит сколько другой пользователь заказал определённых книг и их стоимость по отдельности.
\end{enumerate}


\paragraph{Вариант использования «Редактирование информации о книге сотрудником или администратором»}

Заинтересованные лица и их требования: пользователь авторизованный на аккаунте с ролью сотрудника или администратора желает изменить информацию об одной из книг.

Предусловие: пользователь авторизован, его аккаунт имеет роль сотрудника или администратора и открыта главная страница сайта.

Постусловие: пользователь изменил информацию о книге.

Основной успешный сценарий:

\begin{enumerate}
	\item Пользователь нажимает "Редактировать".
	\item Пользователь спускается в открывшуюся форму редактирования и вносит требуемые изменения.
	\item Пользователь нажимает кнопку "Редактировать" и внесённые изменения сервер записывает в базу данных.
\end{enumerate}


\paragraph{Вариант использования «Удаление книги сотрудником или администратором»}

Заинтересованные лица и их требования: пользователь авторизованный на аккаунте с ролью сотрудника или администратора желает удалить одну из книг.

Предусловие: пользователь авторизован, его аккаунт имеет роль сотрудника или администратора и открыта главная страница сайта.

Постусловие: пользователь удаляет одну из книг.

Основной успешный сценарий:

\begin{enumerate}
	\item Пользователь нажимает "Редактировать".
	\item Пользователь спускается в открывшуюся форму редактирования и вносит и нажимает кнопку "Удалить книгу".
	\item На сайте всплывает уведомление предупреждающее об удалении книги.
	\item Пользователь нажимает кнопку "Да" и подтверждает удаление.
	\item API отправляет запрос на сервер и удаляет книгу из базы данных.
\end{enumerate}


\paragraph{Вариант использования «Добавление книги сотрудником или администратором»}

Заинтересованные лица и их требования: пользователь авторизованный на аккаунте с ролью сотрудника или администратора желает добавить в систему новую книгу.

Предусловие: пользователь авторизован, его аккаунт имеет роль сотрудника или администратора и открыта главная страница сайта.

Постусловие: пользователь вносит в базу данных новую книгу и информацию о ней.

Основной успешный сценарий:

\begin{enumerate}
	\item Пользователь нажимает на кнопку "Добавить книгу" в форме панели сотрудника или администратора.
	\item Приложение открывает форму для заполнения данными о книге.
	\item Заполнив необходимые поля пользователь нажимает кнопку "Сохранить".
	\item Приложение через API загружает книгу и информацию о ней на сервер.
	\item Приложение отображает уведомление об успешном добавлении и книга появляется в общем каталоге.
\end{enumerate}

\paragraph{Вариант использования «Регистрация пользователя»}

Заинтересованные лица и их требования: неавторизованный пользователь желает зарегестрироваться на сайте.

Предусловие: авторизация не была произведена и открыта главная страница сайта.

Постусловие: пользователь создаёт аккаунт, с помощью которого, впоследствии, можно будет производить вход. 

Основной успешный сценарий:

\begin{enumerate}
	\item Пользователь нажимает кнопку "Войти".
	\item В появившемся модальном окне пользователь нажимает кнопку "Регистрация".
	\item Приложение отображает форму с полями для регистрации.
	\item Пользователь заполняет все поля и нажимает "Регистрация".
	\item Приложение отправляет данные на сервер через API и создаёт учётную запись с ролью покупателя.
	\item Пользователь видит уведомление об успешной регистрации.
\end{enumerate}

\paragraph{Вариант использования «Авторизация пользователя»}

Заинтересованные лица и их требования: неавторизованный пользователь желает войти в свою учётную запись на сайте.

Предусловие: пользователь не авторизован, учётная запись уже существует и открыта главная страница сайта.

Постусловие: пользователь входит в свою учётную запись и открывает новый функционал.

Основной успешный сценарий:

\begin{enumerate}
	\item Пользователь нажимает кнопку "Войти".
	\item В появившемся модальном окне приложение отображает форму с полями для авторизации.
	\item Пользователь вводит данные и нажимает "Войти".
	\item Приложение отправляет запрос на сервер через API и проверяет правильность введённых данных.
	\item Пользователь видит уведомление об успешной авторизации. 
	
\end{enumerate}


\subsubsection{Требования к интерфейсу}

Графический интерфейс реализуется с использованием современных веб-технологий HTML, CSS, JavaScript, и предоставляет пользователям интуитивно понятный и удобный интерфейс для взаимодействия с системой. 

Обязательные элементы интерфейса:
\begin{enumerate}
	\item Сетка книг на странице.
	\item Шапка с названием и кнопками авторизации и просмотра заказов.		
	\item Модальные окна регистрации и авторизации пользователей.	
	\item Корзина для совершения покупок.	
	\item Модальное окно со списком заказов.
	\item Модальные окна с подробной информацией о книгах.
	\item Панели сотрудника и администратора для управления магазином.
\end{enumerate}
Дополнительные требования:
\begin{enumerate}
	\item Интерфейс должен корректно масштабироваться при изменении размеров окна.	
	\item Четкие контрастные цвета для текста и фона.
	\item Отсутствие перегруженности элементами на странице.
	\item Интерфейс должен обеспечивать простоту эксплуатации без необходимости использования сторонних инструментов.
\end{enumerate}

\subsubsection{Нефункциональные требования к программной системе}

\paragraph{Требования к надежности}

В процессе работы программно-информационной системы управления книжным магазином могут возникать следующие аварийные ситуации:

\begin{itemize}
	\item потеря доступа к сети Интернет, вызванная сменой типа подключения (Wi-Fi, мобильный интернет) или отсутствием связи в определённой зоне;
	\item принудительная остановка выполнения приложения в браузере пользователя (например, из-за сбоя оборудования или программного обеспечения);
	\item сбой в работе сервера, вызванный программными или аппаратными неисправностями.
\end{itemize}

Для обеспечения надежности серверной части системы рекомендуются следующие меры:
\begin{itemize}
	\item размещение серверных компонентов на выделенных серверах в сертифицированных дата-центрах с гарантией SLA > 99,8\%;
	\item регулярное обновление операционной системы сервера и обеспечение работы систем резервного копирования;
	\item наличие источников бесперебойного питания и резервных каналов связи для предотвращения сбоев в работе системы.
\end{itemize}


\paragraph{Требования к программному обеспечению}

Для реализации программной системы должны быть использованы следующие языки программирования:

\begin{itemize}
	\item Python — серверная часть, веб-приложение;
	\item JavaScript (React) — клиентская часть веб-приложения;
	\item SQL — язык структурированных запросов для работы с PostgreSQL.
\end{itemize}

Для работы клиентской части требуется современный веб-браузер с поддержкой HTML5 и CSS3.
Для работы серверных компонентов необходима операционная система Windows Server 2019 с установленной PostgreSQL и средой выполнения Python 3.11.

\paragraph{Требования к аппаратному обеспечению}

Для сервера необходим центральный процессор с количеством ядер от 4 и выше с тактовой частотой не менее 2.0 ГГц. Объем оперативной памяти должен составлять не менее 16 ГБ.
Требование к скорости интернет-соединения — 50 Мбит/с и выше.

\subsection{Требования к оформлению документации}

Разработка программной документации и программного изделия должна производиться согласно ГОСТ 19.102-77 и ГОСТ 34.601-90. Единая система программной документации.
