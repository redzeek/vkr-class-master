\section{Техническое задание}
\subsection{Основание для разработки}

Основанием для разработки является задание на выпускную квалификационную работу бакалавра.

 Полное наименование системы: «Программно-информационная система управления книжным магазином».

\subsection{Цель и назначение разработки}

Разрабатываемая программно-информационная система предназначена для управления ассортиментом книг, обработки заказов, предоставления удобного интерфейса покупателям и сотрудникам.

Программа ориентирована на предпринимателей, планирующих в будущем выходить на всероссийский рынок.

Задачами данной разработки являются:
\begin{enumerate}
\item Создание серверной части на Flask для обработки запросов и взаимодействия с базой данных PostgreSQL.
\item Разработка клиентской части на HTML, CSS и JavaScript для отображения книг, корзины и управления заказами.
\item Реализация системы аутентификации и авторизации пользователей с разными ролями (клиент, сотрудник, администратор).
\item Интеграция механизмов поиска и пагинации для удобного просмотра каталога книг.
\end{enumerate}


\subsection{Требования к программной системе}


\subsubsection{Требования к данным программной системы}


На рисунке 2.1 представлена модель данных программной системы в виде UML-диаграммы сущность-связь.


\begin{figure}[H]
	\centering
	\includegraphics[width=0.7\linewidth]{"images/Модель_данных"}
	\caption{Модель данных}
	\label{fig:--}
\end{figure}

Входными данными системы являются:
\begin{itemize}
	\item сведения о книгах;
	\item параметры корзины
	\item учётные данные пользователя
\end{itemize}

Выходными данными являются:
\begin{itemize}
	\item идентификаторы оформленных заказов
	\item уведомления о статусе заказов
\end{itemize}



Программно-информационная система должна обеспечивать хранение и обновление данных о книгах, пользователях и заказах.

\subsubsection{Функциональные требования к программной системе}

Разрабатываемая программно-информационная система книжного магазина предусматривает три роли пользователей: покупатели, сотрудники и администраторы. Каждой роли доступны определенные функции.

Для всех пользователей:

\begin{enumerate}
	\item Просмотр каталога книг:
	\begin{itemize}
		\item отображение списка книг с возможностью сортировки и фильтрации;
		\item поиск книг по названию, автору или жанру.
	\end{itemize}
	\item Просмотр деталей книги:
	\begin{itemize}
		\item информация о названии, авторе, цене, наличии на складе и описании.
	\end{itemize}
	\item Авторизация и регистрация:
	\begin{itemize}
		\item вход в систему под своей учетной записью;
		\item создание нового аккаунта.
	\end{itemize}
\end{enumerate}

Для зарегистрированных покупателей:
\begin{enumerate}
	\item Оформление заказов:
	\begin{itemize}
		\item подтверждение заказа с указанием итоговой суммы;
	\end{itemize}
	\item Просмотр истории заказов:
	\begin{itemize}
		\item отслеживание своих заказов;
		\item отслеживание текущего статуса заказов.
	\end{itemize}
\end{enumerate}

Для сотрудников:
\begin{enumerate}
	\item Управление книгами:
	\begin{itemize}
		\item добавление новых книг в каталог;
		\item редактирование информации о существующих книгах;
		\item удаление книг из каталога.
	\end{itemize}
	\item Управление заказами:
	\begin{itemize}
		\item просмотр всех заказов;
		\item изменение статуса заказов.
	\end{itemize}
\end{enumerate}

Для администраторов:
\begin{enumerate}
	\item Управление пользователями:
	\begin{itemize}
		\item назначение ролей;
		\item просмотр списка всех пользователей.
	\end{itemize}
	\item Полный доступ к функциям сотрудников:
	\begin{itemize}
		\item просмотр всех заказов;
		\item все возможности сотрудников, включая управление книгами и заказами.
	\end{itemize}
\end{enumerate}

На рисунке 2.2 в виде диаграммы прецедентов представлены функциональные требования к системе, доступные для всех категорий пользователей.

На рисунке 2.3 представлены дополнительные функциональные требования к системе для авторизованных.

На рисунке 2.4 представлены дополнительные функциональные требования к системе для пользователей с ролью сотрудника.

На рисунке 2.5 представлены дополнительные функциональные требования к системе для пользователей с ролью администратора.

\begin{figure}[H]
	\centering
	\includegraphics[width=0.7\linewidth]{"images/Все_пользователи"}
	\caption{Диаграмма прецедентов для неавторизованных пользователей}
	\label{fig:--}
\end{figure}

\begin{figure}[H]
	\centering
	\includegraphics[width=0.7\linewidth]{"images/Авторизованные"}
	\caption{Диаграмма прецедентов для авторизованных пользователей}
	\label{fig:--}
\end{figure}

\begin{figure}[H]
	\centering
	\includegraphics[width=0.7\linewidth]{"images/Сотрудники"}
	\caption{Диаграмма прецедентов для пользователей с ролью сотрудника}
	\label{fig:--}
\end{figure}

\begin{figure}[H]
	\centering
	\includegraphics[width=0.7\linewidth]{"images/Администраторы"}
	\caption{Диаграмма прецедентов для пользователей с ролью администратора}
	\label{fig:--}
\end{figure}


\paragraph{Вариант использования «Просмотр каталога книг»}

Заинтересованные лица и их требования: покупатель, гость сайта, желает ознакомиться с ассортиментом книг интернет-магазина.

Предусловие: открыта главная страница сайта.

Постусловие: пользователь видит список книг.

Основной успешный сценарий:

\begin{enumerate}
	\item Пользователь заходит на сайт
	\item Приложение загружает список книг с пагинацией (по 8 книг на страницу).
\end{enumerate}


\paragraph{Вариант использования «Поиск книг»}

Заинтересованные лица и их требования: пользователь желает найти конкретную книгу.

Предусловие: открыта главная страница сайта.

Постусловие: пользователь находит или не находит нужную ему книгу.

Основной успешный сценарий:

\begin{enumerate}
	\item Пользователь пишет полное название книги или её часть в поле поиска.
	\item Приложение отправляет запрос к API. 
	\item Сервер возвращает книги, название которых соответствуют поисковому запросу.
	\item Пользователь может переключать страницы с результатами.
\end{enumerate}


\paragraph{Вариант использования «Просмотр деталей книги» }

Заинтересованные лица и их требования: покупатель, желающий узнать подробности о книге.

Предусловие: открыта главная страница сайта.

Постусловие: пользователь видит полную информацию о книге.

Основной успешный сценарий:

\begin{enumerate}
	\item Пользователь нажимает на кнопку "Подробнее".
	\item Приложение отправляет запрос к API для получения полной информации о книге.
	\item Открывается модальное окно с деталями: автор, цена, описание, жанры и т.д.
	\item Пользователь может закрыть окно или добавить книгу в корзину.
\end{enumerate}


\paragraph{Вариант использования «Добавление книги в корзину»}

Заинтересованные лица и их требования: покупатель, который авторизован на сайте, желает добавить книгу в корзину.

Предусловие: пользователь авторизован и открыта главная страница сайта.

Постусловие: книга добавлена в корзину.

Основной успешный сценарий:

\begin{enumerate}
	\item Пользователь нажимает кнопку "В корзину" на карточке книги.
	\item Если пользователь не авторизован, появляется уведомление с предложением войти.
	\item Книга добавляется в корзину (локальное хранилище и состояние UI обновляются).
	\item Кнопка меняется на "В корзине" и становится неактивной.
	\item Появляется возможность нажать кнопку "Оформить заказ"
	и отображается стоимость корзины.
\end{enumerate}


\paragraph{Вариант использования «Управление корзиной»}

Заинтересованные лица и их требования: авторизованный покупатель желает изменить свою корзину.

Предусловие: в корзине есть хотя бы один товар и открыта главная страница сайта.

Постусловие: состояние корзины обновлено.

Основной успешный сценарий:

\begin{enumerate}
	\item Пользователь спускается в корзину.
	\item Пользователь может увеличить или уменьшить количество выбранных книг.
	\item Пользователь может выборочно удалить книги из корзины.
	\item Обновляется общая стоимость заказа.
	\item Если товара нет в достаточном количестве, отображается предупреждение.
	\item При соблюдении условий становится доступна кнопка "Оформить заказ".
\end{enumerate}


\paragraph{Вариант использования «Оформление заказа»}

Заинтересованные лица и их требования: авторизованный покупатель, который желает заказать книги из корзины.

Предусловие: открыта главная страница сайта, есть книги в корзине и выбрано имеющееся на складе их количество.

Постусловие: заказ создан, корзина очищена.

Основной успешный сценарий:

\begin{enumerate}
	\item Пользователь нажимает "Оформить заказ".
	\item Приложение отправляет на сервер данные о заказе.
	\item Сервер проверяет наличие товаров и создает заказ.
	\item Пользователь получает уведомление об успешном оформлении с номером заказа и деталями.
	\item Корзина очищается.
\end{enumerate}


\paragraph{Вариант использования «Просмотр истории заказов»}

Заинтересованные лица и их требования: авторизованный покупатель, желает посмотреть на историю заказов.

Предусловие: пользователь авторизован и открыта главная страница сайта

Постусловие: отображается список заказов.

Основной успешный сценарий:

\begin{enumerate}
	\item Пользователь нажимает "Мои заказы".
	\item Приложение загружает список заказов через API.
	\item При наличии заказов для каждого отображаются номер, дата, статус, сумма.
\end{enumerate}


\paragraph{Вариант использования «Просмотр подробностей заказа»}

Заинтересованные лица и их требования: авторизованный покупатель желает подробнее изучить свой заказ.

Предусловие: пользователь авторизован, на главной странице и имеет хотя бы один заказ.

Постусловие: пользователь узнаёт подробности своего заказа.

Основной успешный сценарий:

\begin{enumerate}
	\item Пользователь нажимает "Мои заказы".
	\item Приложение загружает список заказов через API.
	\item Пользователь нажимает на один из своих заказов.
	\item При наличии этих книг в системе пользователь увидит сколько он заказал определённых книг и их стоимость по отдельности.
\end{enumerate}


\paragraph{Вариант использования «Просмотр истории заказов сотрудником или администратором»}

Заинтересованные лица и их требования: пользователь авторизованный на аккаунте с ролью сотрудника или администратора желает посмотреть список заказов всех пользователей.

Предусловие: пользователь авторизован, его аккаунт имеет роль сотрудника или администратора и открыта главная страница сайта.

Постусловие: пользователь видит список всех заказов.

Основной успешный сценарий:

\begin{enumerate}
	\item Пользователь нажимает "Все заказы".
	\item Приложение загружает список заказов через API.
	\item При наличии заказов для каждого отображаются номер, дата, статус, сумма.
\end{enumerate}


\paragraph{Вариант использования «Просмотр подробностей заказа сотрудником или администратором»}

Заинтересованные лица и их требования: пользователь авторизованный на аккаунте с ролью сотрудника или администратора желает посмотреть подробности заказа одного из пользователей.

Предусловие: пользователь авторизован, его аккаунт имеет роль сотрудника или администратора и открыта главная страница сайта.

Постусловие: пользователь узнаёт подробности заказа одного из пользователей.

Основной успешный сценарий:

\begin{enumerate}
	\item Пользователь нажимает "Все заказы".
	\item Приложение загружает список заказов через API.
	\item Пользователь нажимает на один из заказов.
	\item При наличии этих книг в системе пользователь увидит сколько другой пользователь заказал определённых книг и их стоимость по отдельности.
\end{enumerate}


\subsubsection{Требования к интерфейсу}

Графический интерфейс реализуется с использованием библиотеки Tkinter и предоставляет пользователю простые и интуитивные средства для взаимодействия с системой.

Обязательные элементы интерфейса:
\begin{enumerate}
	\item Каталог книг на странице.
	\item Поиск книг через строку поиска и смену страниц каталога.		
	\item Окно регистрации и авторизации пользователей.	
	\item Корзина для совершения покупок.	
	\item Окно со списком заказов.
	\item Панели сотрудника и администратора для управления магазином.
\end{enumerate}
Дополнительные требования:
\begin{enumerate}
	\item Интерфейс должен корректно масштабироваться при изменении размеров окна.	
	\item Четкие контрастные цвета для текста и фона.
	\item Отсутствие перегруженности элементами на странице.
	\item Интерфейс должен обеспечивать простоту эксплуатации без необходимости использования сторонних инструментов.
\end{enumerate}


\subsection{Требования к оформлению документации}

Разработка программной документации и программного изделия должна производиться согласно ГОСТ 19.102-77 и ГОСТ 34.601-90. Единая система программной документации.
