\section*{ЗАКЛЮЧЕНИЕ}
\addcontentsline{toc}{section}{ЗАКЛЮЧЕНИЕ}

В результате выполнения выпускной квалификационной работы была разработана программно-информационная система управления книжным магазином в виде веб-приложения с клиент-серверной архитектурой. Использование современных веб-технологий позволило создать гибкое, надёжное и масштабируемое решение, отвечающее требованиям автоматизации процессов книжного магазина.

Разработанное веб-приложение обеспечивает удобный доступ к каталогу книг, оформление заказов и управление данными для трёх ролей пользователей: покупатель, сотрудник и администратор. Система предоставляет адаптивный пользовательский интерфейс, обеспечивающий комфортное использование на различных устройствах, и эффективно обрабатывает запросы благодаря REST API.

Основные результаты работы:

\begin{enumerate}
\item Проведён анализ предметной области интернет-торговли книгами, определены ключевые сущности системы и сформулированы требования к системе.
\item Разработана концептуальная модель системы, включая модель данных и архитектуру клиент-серверного взаимодействия.
\item Спроектирована программная система. Разработана архитектура, включающая серверные и клиентские модули. REST API обеспечивает взаимодействие. Flask обрабатывает запросы и взаимодействует с PostgreSQL через psycopg2. Интерфейс адаптивен и поддерживает все функции.
\item Реализовано и протестировано веб-приложение. Серверная часть обеспечивает обработку запросов, клиентская часть -- динамический интерфейс с адаптивным дизайном. Проведено системное тестирование, подтвердившее работоспособность функций.
\end{enumerate}

Все требования, объявленные в техническом задании, были полностью реализованы, все задачи, поставленные в начале разработки проекта, были также решены.
