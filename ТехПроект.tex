\section{Технический проект}
\subsection{Общая характеристика организации решения задачи}

Программно-информационная система представляет собой современное веб-приложение, предназначенное для автоматизации управления книжным магазином. Система разработана с использованием модульной архитектуры, что позволяет легко адаптировать её к потребностям малого и среднего бизнеса, а также расширять функционал при необходимости.

В основе системы лежит серверная часть, реализованная на Flask с использованием RESTful API. Это обеспечивает быстрое и удобное взаимодействие между клиентской частью и сервером. Веб-интерфейс построен с использованием HTML, CSS и JavaScript, что делает систему доступной в любом современном браузере. База данных PostgreSQL выступает в качестве хранилища данных, а библиотека psycopg2 используется для взаимодействия с ней.

Система предоставляет удобные возможности для всех категорий пользователей. Покупатели могут просматривать каталог книг, использовать поиск, корзину для оформления заказов и отслеживать их статус. Также доступна страница книги с подробной информацией, включая описание и дополнительные характеристики.

Для сотрудников магазина предусмотрен интерфейс, позволяющий редактировать информацию о книгах, изменять их наличие, а также обновлять статусы заказов покупателей. Администраторы имеют доступ ко всему функционалу сотрудников, а также к управлению ролями пользователей. Это позволяет быстро изменять права доступа, добавлять новых сотрудников или ограничивать доступ к определённым функциям.

Система поддерживает аутентификацию и авторизацию пользователей, обеспечивая безопасное использование. Реализована возможность работы с несколькими ролями: покупатель, сотрудник и администратор, что делает её гибкой и подходящей для различных сценариев использования.

Ключевая особенность системы – её масштабируемость. Архитектура позволяет легко добавлять новые функции, такие как поддержка скидок, интеграция с платёжными системами или создание аналитических отчётов о продажах. Также предусмотрена возможность адаптации интерфейса под мобильные устройства, что обеспечивает доступность для пользователей с разных платформ.

Применение системы эффективно в книжных магазинах, где требуется централизованное управление ассортиментом и заказами. Её также можно модифицировать для других сфер розничной торговли. Перспективы развития включают добавление системы промокодов, расширение функционала аналитики и поддержку мультиязычного интерфейса, что сделает её полезным инструментом для международных пользователей.

Эта система сочетает простоту в использовании и высокую гибкость настройки, предоставляя удобные инструменты для управления книжным магазином.

\subsection{Обоснование выбора технологии проектирования}

Выбор технологий, языков программирования и архитектурных решений для реализации программно-информационной системы обусловлен совокупностью факторов, направленных на обеспечение высокой гибкости, надёжности и простоты сопровождения программного продукта. Используемые для создания программно-информационной системы языки и технологии отвечают современным практикам разработки, позволяют достичь высокой производительности и отказоустойчивости программы.

\subsubsection{Язык программирования Python}

В качестве основного языка программирования выбран Python, благодаря его сочетанию выразительности, гибкости и обширной поддержки со стороны сообщества разработчиков. Python — это высокоуровневый, интерпретируемый язык, активно применяющийся как в образовательных, так и в промышленных проектах. Основные причины выбора языка заключаются в следующем:
\begin{enumerate}
	\item Простой и интуитивно понятный синтаксис значительно сокращает порог вхождения и снижает количество потенциальных ошибок при написании кода. Это особенно важно в условиях ограниченного времени на разработку и тестирование, а также при передаче проекта на сопровождение.
	\item Поддержка нескольких парадигм программирования, включая объектно-ориентированную, процедурную и функциональную, делает Python универсальным инструментом. Это позволяет организовать код в соответствии с принципами модульности, инкапсуляции и повторного использования.
	\item Обширная стандартная библиотека и внешняя экосистема обеспечивают доступ к готовым модулям для сериализации, построения интерфейса, анализа синтаксических деревьев, многопоточности и многого другого. Это существенно ускоряет разработку и упрощает реализацию сложных функций.
	\item Кроссплатформенность языка позволяет запускать приложение на операционных системах Windows, Linux и macOS без необходимости адаптации кода под конкретную платформу. Таким образом, обеспечивается максимальная универсальность и доступность системы для пользователя.	
\end{enumerate}
Таким образом, Python представляет собой оптимальное решение для реализации проекта, сочетающее в себе простоту, мощь и гибкость, что делает его незаменимым инструментом в учебных и практических задачах программной инженерии.

\subsubsection{Графический интерфейс с использованием HTML и CSS}

Для построения графического интерфейса выбран HTML — это язык разметки, который используется для создания структуры веб-страниц. Он позволяет определить, какие элементы (например, текст, изображения, формы, кнопки) будут отображаться на странице и как они будут структурированы. HTML является основой любого веб-интерфейса, предоставляя платформу для добавления контента и создания навигации.
Основные преимущества использования HTML:
\begin{enumerate}
	\item Простая и понятная технология, которая легко осваивается и используется для создания структурированных веб-страниц. Это позволяет быстро разрабатывать интерфейсы и адаптировать их под требования проекта. Разработчики могут сосредоточиться на бизнес-логике, не тратя много времени на изучение сложных технологий.
	\item HTML поддерживается всеми современными браузерами (Chrome, Firefox, Safari, Edge и другие), что гарантирует корректное отображение страницы на разных устройствах. Это важно для обеспечения доступности системы для пользователей с разными предпочтениями по браузерам и операционным системам.
	\item HTML предоставляет широкий набор тегов для создания различных элементов интерфейса, таких как текстовые блоки, изображения, ссылки, формы и кнопки. Это позволяет создавать подробные страницы каталогов книг, карточки товаров, страницы заказов и другие компоненты, нужные для книжного магазина.
	\item HTML позволяет разделить страницу на логические блоки, что упрощает разработку и поддержку интерфейса. Например, каждый раздел страницы (каталог книг, корзина, информация о книге) может быть реализован в виде отдельного блока, который можно редактировать или заменять без затрагивания других частей страницы.
\end{enumerate}
CSS — это язык стилей, который используется для оформления внешнего вида веб-страниц, созданных на HTML. С помощью CSS можно задать цвета, шрифты, размеры, расположение элементов, а также добавлять анимации и переходы. CSS позволяет сделать интерфейс не только функциональным, но и эстетически привлекательным и удобным для пользователей.
\begin{enumerate}
	\item CSS позволяет легко управлять внешним видом интерфейса. С помощью CSS можно изменять цвета фона, шрифт текста, отступы, выравнивание элементов и другие аспекты оформления. Это даёт возможность создавать стильные и гармоничные страницы, которые соответствуют фирменному стилю магазина и предпочтениям пользователей.
	\item Одной из ключевых особенностей CSS является возможность реализации адаптивного дизайна, который автоматически подстраивается под различные размеры экранов устройств. Это особенно важно для обеспечения удобства использования интерфейса как на настольных компьютерах, так и на мобильных устройствах, таких как смартфоны и планшеты. Используя медиа-запросы, можно настроить отображение контента в зависимости от размера экрана, обеспечив оптимальное восприятие интерфейса на разных устройствах.
	\item Разделение структуры (HTML) и оформления (CSS) позволяет легко изменять внешний вид приложения, не затрагивая его функциональность. Это упрощает поддержку и улучшение интерфейса, так как дизайнеры и разработчики могут работать над внешним видом страницы, не изменяя код её структуры.
	\item С помощью CSS можно создавать плавные анимации и переходы между различными состояниями элементов, например, при наведении мыши на кнопки или изменения состояния корзины. Это улучшает пользовательский опыт, делая взаимодействие с интерфейсом более динамичным и приятным.
\end{enumerate}


Выбор HTML и CSS для создания графического интерфейса системы управления книжным магазином обоснован их простотой, гибкостью, поддержкой кросс-браузерности и возможностью создания адаптивного дизайна. Эти технологии позволяют создать интуитивно понятный, функциональный и привлекательный интерфейс, который будет удобен для всех категорий пользователей и обеспечит хорошую производительность системы.