\section{Анализ предметной области}
\subsection{Понятие и ключевые аспекты интернет-торговли}

Интернет-торговля (или электронная коммерция) – это форма коммерческой деятельности, при которой сделки купли-продажи товаров, услуг или цифровых продуктов осуществляются через интернет с использованием электронных платежных систем и цифровых коммуникационных каналов \cite{1}.
Из этого термина можно выделить две основные концепции интернет-торговли:

\begin{enumerate}
	\item Широкий подход – интернет-торговля рассматривается как синоним электронной коммерции, охватывающей все аспекты бизнеса: маркетинг, логистику, платежи, CRM-системы и послепродажное обслуживание.
	\item Узкий подход – интернет-торговля трактуется как частный случай электронной коммерции, ограниченный дистанционной продажей товаров через интернет.
\end{enumerate}

В статье «Развитие электронной торговли в Российской Федерации» С.С. Корнева выделяет девять ключевых форм электронной коммерции, классифицируемых по типу взаимодействия участников сделки \cite{2}.:
\begin{itemize}
	\item B2B (Business-to-Business) – сделки между коммерческими организациями. Включает корпоративные электронные рынки и внутренние системы управления предприятием. На эту модель приходится наибольший объём электронных сделок;
	\item B2C (Business-to-Consumer) – продажи товаров и услуг от бизнеса конечным потребителям. К такой форме относятся интернет-магазины (Ozon, Citilink) и разного рода онлайн-услуги (образование, страхование, банкинг);
	\item B2G (Business-to-Government) – государство предоставляет бизнесу услуги (налоги, лицензирование, закупки) в электронном формате;
	\item C2C (Consumer-to-Consumer) – торговля между частными лицами через онлайн-платформы. Яркими представителями такой интернет-торговли являются доски объявлений (Avito);
	\item C2B (Consumer-to-Business) – частные лица предлагают товары/услуги компаниям. К таким, например, относятся краудсорсинговые платформы (Planeta, Kickstarter);
	\item C2G (Consumer-to-Government) – частные лица участвуют в сделке с правительственной структуре, производят выплаты по счетам;
	\item G2C (Government-to-Consumer) – взаимодействие государства с гражданами через цифровые сервисы, например, выплаты пособий (Госуслуги);
	\item G2B (Government-to-Business) – государство предоставляет бизнесу услуги (налоги, лицензирование) в электронном формате;
	\item G2G (Government-to-Government) – электронный документооборот между госучреждениями; 
 \end{itemize}
 
На данный момент доминирующими формами являются B2B и B2G \cite{2}.
Интернет-торговля обладает рядом значительных преимуществ по сравнению с традиционной розничной торговлей. Эти преимущества касаются как бизнеса, так и потребителей, а также способствуют развитию экономики в целом.

Для бизнеса можно выделить следующие основные выгоды:
 \begin{itemize}
 	\item глобальный охват рынка. Интернет-магазин доступен круглосуточно из любой точки мира, что позволяет привлекать клиентов за пределами локального рынка;
 	\item снижение операционных затрат. Отсутствие расходов на аренду торговых площадей, коммунальные услуги и большое количество персонала. Автоматизация процессов (учёт товаров, обработка заказов, CRM-системы) сокращает издержки;
 	\item упрощение логистики и автоматизация. Интеграция с курьерскими службами и маркетплейсами ускоряет доставку.
 \end{itemize}
 
 Для потребителей:
 \begin{itemize}
 	\item удобство и экономия времени. Покупки в любое время суток без необходимости посещения магазинов, доступ к широкому ассортименту товаров и сравнение цен в несколько кликов;
 	\item возможность возврата и гарантийного обслуживания. Законодательство о дистанционной торговле защищает права покупателей (например, право на возврат в течение 7 дней) \cite{3}, упрощённые процедуры обмена и возврата через онлайн-поддержку;
 	\item доступ к скидкам и акциям. Купоны, кэшбэк, программы лояльности и подписки (например, Wildberries Premium), алгоритмы персонализированных предложений на основе истории покупок.
 \end{itemize}
 
 Для экономики и общества:
 \begin{itemize}
 	\item стимулирование цифровизации. Развитие ИТ-инфраструктуры, платежных систем и логистических сервисов, создание новых рабочих мест (разработчики, маркетологи, курьерские службы);
 	\item поддержка малого бизнеса. Низкий порог входа для стартапов (можно начать с маркетплейсов или соцсетей), Доступ к инструментам продвижения (таргетированная реклама, SEO).
 \end{itemize}
 
 Интернет-торговля предлагает выгоды для всех участников рынка. Её развитие продолжает трансформировать розничную торговлю, делая её более эффективной и клиентоориентированной.
 
\subsubsection{Интернет-магазины как ключевая форма интернет-торговли}

Интернет-магазин — это виртуальная торговая площадка, функционирующая на базе интернет-технологий, где покупатели могут выбирать, заказывать и оплачивать товары или услуги онлайн. В отличие от традиционной розничной торговли, интернет-магазин не требует физического присутствия покупателя и продавца в одном месте, что значительно расширяет географию продаж и снижает издержки.

Ключевые особенности интернет-магазинов:
 \begin{itemize}
 	\item электронная витрина — веб-сайт с каталогом товаров, подробными описаниями, изображениями и ценами;
 	\item торговая система — автоматизированные процессы оформления заказа, оплаты и доставки;
 	\item корзина покупок — функционал, позволяющий пользователю собирать выбранные товары и редактировать заказ перед оплатой.
 \end{itemize}
 
 Интернет-магазины представляют собой динамично развивающийся сегмент электронной коммерции, сочетающий технологические инновации с удобством для потребителей. Их дальнейший рост будет зависеть от адаптации к изменяющимся рыночным условиям и внедрения новых цифровых решений \cite{4}.
 
\subsubsection{Интернет-торговля в современном бизнесе}

Интернет-торговля кардинально изменила подходы к ведению бизнеса, создав новые возможности для компаний различных масштабов. Её роль в современной экономике невозможно переоценить, так как она позволяет выходить на глобальные рынки, оптимизировать издержки и выстраивать эффективные коммуникации с потребителями \cite{5}.

Одним из ключевых преимуществ интернет-торговли для бизнеса является доступ к международным рынкам без необходимости открытия физических представительств. Российские компании, такие как Wildberries и Ozon, успешно используют этот потенциал, расширяя своё присутствие за рубежом. Кроме того, онлайн-формат торговли значительно снижает операционные затраты за счёт экономии на аренде помещений, содержании персонала и других традиционных статьях расходов. Автоматизация бизнес-процессов с помощью CRM и ERP-систем дополнительно повышает эффективность управления.

Гибкость интернет-торговли позволяет компаниям быстро адаптироваться к изменениям спроса, тестировать новые продукты и масштабировать бизнес с минимальными рисками. Персонализация взаимодействия с клиентами, основанная на анализе данных, даёт возможность предлагать индивидуальные решения и повышать лояльность аудитории. Круглосуточная доступность онлайн-магазинов обеспечивает стабильный поток продаж без временных ограничений.

Влияние интернет-торговли на экономику проявляется в её растущем вкладе в ВВП. Согласно исследованиям, объём рынка электронной коммерции в России увеличился с 1,7 трлн рублей в 2017 году до 8 трлн рублей в 2020 году \cite{6}. Этот рост сопровождается созданием новых рабочих мест в сферах логистики, ИТ и цифрового маркетинга, а также стимулированием малого бизнеса, для которого онлайн-платформы стали доступным инструментом выхода на рынок.

Однако развитие интернет-торговли сопряжено с рядом вызовов. Высокая конкуренция, особенно со стороны международных гигантов, таких как Amazon и Aliexpress, требует от компаний инновационных решений. Логистические сложности, особенно в удалённых регионах, и недостаточная развитость инфраструктуры также остаются серьёзными барьерами. Правовые риски, связанные с отсутствием единого регулирования, и угрозы кибербезопасности добавляют неопределённости в работу онлайн-бизнеса.

Перспективы развития интернет-торговли связаны с внедрением передовых технологий, таких как дополненная и виртуальная реальность для виртуальных примерок, а также голосовая коммерция. Важную роль играет государственная поддержка, включая меры по легализации отрасли, развитию цифровой инфраструктуры и созданию благоприятных налоговых условий.
Таким образом, интернет-торговля продолжает трансформировать бизнес-среду, предлагая компаниям инструменты для роста и повышения конкурентоспособности. Для устойчивого развития отрасли необходимо решать существующие задачи, включая вопросы регулирования, логистики и технологий, чтобы максимально реализовать её потенциал в экономике будущего \cite{5}.



\subsubsection{Интернет-торговля для потребителя}

Интернет-торговля кардинально изменила потребительское поведение, предоставив покупателям новые возможности и изменив их подход к совершению покупок. Как отмечает Воскресенская О.В., онлайн-магазины стали не просто альтернативой традиционной розничной торговле, а основным каналом удовлетворения потребностей современного потребителя \cite{7}.

Онлайн-площадки обеспечивают потребителям круглосуточный доступ к товарам и услугам, позволяя совершать покупки в любое время и из любой точки мира. Это особенно важно в условиях глобализации, когда покупатели могут выбирать товары не только у локальных продавцов, но и у международных поставщиков \cite{8}.

Интернет-магазины предлагают значительно более широкий ассортимент по сравнению с физическими магазинами. Потребители могут легко сравнивать цены, читать отзывы и изучать характеристики товаров перед покупкой.

Просмотр товаров онлайн стал формой развлечения, что увеличивает частоту покупок, а изобилие аналогов учит потребителей более внимательно оценивать качество и стоимость товаров.

Интернет-торговля трансформировала потребительское поведение, сделав покупки более удобными, но и более сложными с точки зрения контроля расходов. Её дальнейшее развитие потребует баланса между удобством, прозрачностью и защитой прав покупателей.

\subsection {История интернет-торговли}
Интернет-торговля зародилась задолго до появления современных онлайн-магазинов. Первые шаги были связаны с электронным обменом данными между компаниями, но настоящий прорыв произошел с распространением интернета.

Сначала онлайн-продажи ограничивались простыми товарами, такими как книги и техника. Постепенно компании начали разрабатывать удобные платформы, системы оплаты и доставки, что сделало интернет-торговлю доступной для массового потребителя.

Со временем ассортимент расширился до практически любых товаров и услуг, а технологии персонализации и маркетинга позволили делать покупки в интернете еще удобнее. Сегодня интернет-торговля — это глобальная индустрия, которая продолжает развиваться, внедряя новые технологии, такие как мобильные платежи, искусственный интеллект и быструю доставку.

\subsubsection{Мировая история развития интернет-торговли}

Развитие интернет-торговли представляет собой многогранную историю технологического прогресса, которая коренным образом изменила мировую экономику и потребительские привычки. Ее становление можно проследить через несколько ключевых этапов.

Первые шаги электронной коммерции были сделаны еще в 1960 году, когда компании IBM и American Airlines разработали инновационную систему для бронирования авиабилетов. Этот проект стал прообразом современных онлайн-транзакций. В 1970-1980-х годах началось активное внедрение систем электронного обмена данными (EDI), что позволило автоматизировать бизнес-процессы в логистике и оптовой торговле. Важным рубежом стал 1989 год, когда интернет приобрел современные черты с появлением стандартов HTTP, что открыло новые возможности для онлайн-коммерции  \cite{9}.

Настоящий прорыв произошел в 1990-х годах, когда в США были сняты ограничения на коммерческое использование интернета. Это десятилетие ознаменовалось появлением пионеров электронной коммерции: в 1994 году была создана первая платежная система CyberCash, в 1995 году Джефф Безос основал Amazon, начавший с продажи книг и превратившийся в крупнейший мировой маркетплейс. Тогда же появился eBay, революционизировавший торговлю через онлайн-аукционы. Эти платформы заложили основы современных моделей B2C и C2C \cite{9}.

\subsubsection{Развитие интернет-торговли в России}

В России интернет-торговля начала развиваться после распада СССР. Первые сделки в 1990-х осуществлялись по системе MOTO (заказ по телефону с последующей почтовой доставкой). На рубеже веков появились первые российские интернет-магазины, к 2020 году весомый процент продаж приходился на маркетплейсы Ozon и Wildberries.

Современный этап характеризуется доминированием маркетплейсов, объединяющих миллионы продавцов и покупателей. Особое значение приобрела мобильная коммерция - по данным исследований, более 60\% покупок сейчас совершается через смартфоны. Российские площадки, успешно выходят на международные рынки, демонстрируя глобализацию электронной торговли \cite{10}.

\subsubsection{Влияние пандемии COVID-19 на интернет-торговлю}

Пандемия COVID-19 стала мощным катализатором для ускоренного развития интернет-торговли, коренным образом изменив потребительские привычки и структуру рынка. По данным исследования, в 2020 году российский рынок онлайн-торговли вырос на 57\%, достигнув объема 2,7 трлн рублей, при этом наиболее значительный рост (на 172\%) наблюдался в сегменте пищевых продуктов. Это было обусловлено вынужденным переходом потребителей на онлайн-шопинг из-за карантинных ограничений \cite{11}.

Ключевые изменения, вызванные пандемией:
\begin{enumerate}
	\item Расширение аудитории:
	\begin{itemize}
		\item прирост пользователей онлайн-торговли составил 10 млн человек, включая ранее неактивную возрастную группу 55+;
		\item увеличилась частота покупок: 45\% респондентов отметили рост повторных заказов.
	\end{itemize}
	
	\item Трансформация логистики:
	\begin{itemize}
		\item бесконтактная доставка (94\% спроса);
		\item самовывоз из магазинов (65\%);
		\item партнерства с курьерскими сервисами и такси для снижения нагрузки на склады.
	\end{itemize}
	
	\item Технологический прорыв:
	\begin{itemize}
		\item 65\% компаний столкнулись с перебоями в ИТ-системах, что стимулировало внедрение решений для масштабирования;
		\item реклама в соцсетях: 20\% бизнесов начали использовать ее впервые.
	\end{itemize}
	
	\item Сдвиг в товарных категориях:
	\begin{itemize}
		\item рост спроса на продукты питания и товары первой необходимости;
		\item снижение продаж одежды и дорогой техники из-за экономической нестабильности.
	\end{itemize}
	
	\item Долгосрочные последствия:
	\begin{itemize}
		\item онлайн-торговля стала дополнять офлайн, а не заменять его: 45\% респондентов считают, что физические магазины сохранят актуальность.
		\item развитие маркетплейсов как ключевого канала сбыта для производителей.
	\end{itemize}
	
\end{enumerate}

COVID-19 не только ускорил цифровизацию торговли, но и выявил ее уязвимости, заставив бизнес адаптироваться.


\subsection{Интернет-торговля книгами}

Интернет-торговля книгами является одним из ключевых направлений развития электронной коммерции. Зарубежный опыт, в частности успех компании Amazon.com, демонстрирует эффективность сочетания инновационных технологий, клиентоориентированного подхода и широкого ассортимента. Amazon.com достигла высоких показателей за счет персонализации сервиса, удобной навигации, системы рекомендаций и быстрой доставки, что позволило ей завоевать доверие миллионов покупателей.

В России данный сегмент также развивается, однако существуют значительные отличия от зарубежных стандартов. Крупнейшие российские книжные интернет-магазины, такие как Ozon, Labirint, Bolero и Книга.ру, пока не могут конкурировать с Amazon.com по уровню сервиса, логистики и масштабам деятельности \cite{12}. Среди основных проблем российского рынка можно выделить:
\begin{itemize}
	\item ограниченный ассортимент электронных книг;
	\item высокие тарифы на доставку, особенно в регионы;
	\item низкую скорость выполнения заказов;
	\item недоверие покупателей к онлайн-продавцам из-за некорректного указания цен и отсутствия товаров в наличии.
\end{itemize}

Для дальнейшего развития интернет-торговли книгами в России необходимо внедрение современных логистических решений, улучшение качества обслуживания клиентов и расширение ассортимента, включая цифровые форматы. Опыт Amazon.com может служить ориентиром для российских компаний, стремящихся повысить конкурентоспособность на международном рынке.