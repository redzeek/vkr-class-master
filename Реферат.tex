\abstract{РЕФЕРАТ}

Объем работы равен \formbytotal{lastpage}{страниц}{е}{ам}{ам}. Работа содержит \formbytotal{figurecnt}{иллюстраци}{ю}{и}{й}, \formbytotal{tablecnt}{таблиц}{у}{ы}{}, \arabic{bibcount} библиографических источников и \formbytotal{числоПлакатов}{лист}{}{а}{ов} графического материала. Количество приложений – 2. Графический материал представлен в приложении А. Фрагменты исходного кода представлены в приложении Б.

Перечень ключевых слов: интернет-магазин, веб-приложение, база данных, Flask, JavaScript, HTML, CSS, PostgreSQL, аутентификация, авторизация, каталог книг, заказы, пользователи, REST API.

Объектом разработки является программно-информационная система управления книжным магазином, реализованная в виде веб-приложения с клиент-серверной архитектурой.

Целью выпускной квалификационной работы является разработка веб-приложения для автоматизации процессов управления книжным магазином, включая просмотр каталога, оформление заказов и администрирование данных.

В процессе разработки были определены основные сущности системы (книги, авторы, жанры, пользователи, заказы), спроектирована реляционная база данных PostgreSQL, реализованы серверная часть на Python с использованием Flask и клиентская часть с применением HTML, CSS и JavaScript. Система поддерживает функционал для трех ролей пользователей: покупатель, сотрудник, администратор. Разработан адаптивный пользовательский интерфейс, обеспечивающий удобство работы на различных устройствах. Проведено тестирование системы для подтверждения ее работоспособности.

\selectlanguage{english}
\abstract{ABSTRACT}
  
The volume of work is \formbytotal{lastpage}{page}{}{s}{s}. The work contains \formbytotal{figurecnt}{illustration}{}{s}{s}, \formbytotal{tablecnt}{table}{}{s}{s}, \arabic{bibcount} bibliographic sources and \formbytotal{числоПлакатов}{sheet}{}{s}{s} of graphic material. The number of applications is 2. The graphic material is presented in annex A. The layout of the site, including the connection of components, is presented in annex B.

List of keywords: online store, web application, database, Flask, JavaScript, HTML, CSS, PostgreSQL, authentication, authorization, book catalog, orders, users, REST API.

Object of development: a software-information system for managing a bookstore, implemented as a web application with a client-server architecture.

Purpose of the final qualifying work: to develop a web application for automating bookstore management processes, including browsing the catalog, placing orders, and administering data.

Development process: the main system entities (books, authors, genres, users, orders) were defined, a relational PostgreSQL database was designed, the server-side was implemented using Python with Flask, and the client-side was developed using HTML, CSS, and JavaScript. The system supports functionality for three user roles: customer, employee, and administrator. An adaptive user interface was created to ensure usability across various devices. System testing was conducted to confirm its functionality.
\selectlanguage{russian}
